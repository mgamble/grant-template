
\section{Resource Sharing Plan}
Albert Einstein College of Medicine is committed to the open and timely dissemination of research outcomes. 
We recognize that innovative new methods, technologies, and strategies may arise during the course of the research proposed here.
The PI of this proposal is aware of and agree to abide by the principles for sharing research resources as described in \enquote{Principles and Guidelines for Recipients of NIH Research Grants and Contracts on Obtaining and Disseminating Biomedical Research Resources.}

\begin{itemize}
	\item Sharing of data generated by this project is an essential part of the proposed activities and will be carried out in several ways. 
	We would wish to make our results available to the community of scientists interested in chromatin biology and histone variants. 
	Also, we would welcome collaboration with others who could make use of our research.
	Our results will be disseminated through timely publication in peer-reviewed scientific journals and presentations at scientific meetings.
	
	\item All cell lines, constructs, and reagents generated in this study will be available to other investigators upon request.
	
	\item All large datasets produced as a part of this proposal (e.g. ChIP-seq, RNA-seq, ATAC-seq, DNA-seq) will be uploaded to the NIH GEO repository and made publicly available. 
\end{itemize}
	
\subsection{Einstein’s policy on data sharing is as follows and will be adhered to as part of the current project}
As Principal Investigator for this application, I am fully cognizant of the requirements for sharing research data and resources (\url{http://grants.nih.gov/grants/policy/data_sharing}). 
The results of the research will be disseminated through the timely publication of our findings in peer-reviewed scientific journals and presentations at scientific meetings. 
Einstein is a signatory to the Uniform Biological Material Transfer Agreement and uses the UBMTA to cover the transfer of biological materials to other academic and non-profit institutions. 
Transfers of biological materials and animals to for-profit entities are also governed by Material Transfer Agreements through Einstein’s Office of Biotechnology. 
Einstein adheres to the NIH's guidelines \enquote{Sharing Biomedical Research Resources: Principles and Guidelines for Recipients of NIH Research Grants and Contracts} and does not impose reach-through royalties on the transfer of materials. 
All Einstein faculty members are required to sign and are bound by the College's \enquote{Policy on Patents and Licensing Agreements.} 
Einstein's policies on patenting and licensing are consistent with the Bayh-Dole Act of 1980. Patenting, licensing, dissemination of research materials and negotiation of sponsored research agreements is handled by Einstein's Office of Biotechnology. 
The Office has an active program of seeking partners in the biotechnology and pharmaceutical industries to further develop and take to market, for the benefit of the public good, Einstein-based technologies with commercial potential.